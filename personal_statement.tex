\documentclass[a4paper, 11pt]{article}
\title{Personal statement for admission to UCL Computer Science PhD programme}
\author{Tim Cowlishaw}
\usepackage{verbatim,titlesec,lmodern,url}
\usepackage[T1]{fontenc}
\date{\today}
\setlength{\parskip}{11pt}
\setlength{\parindent}{0pt}
\begin{document}
\maketitle
Since completing my MSc in Computer Science in 2011, I have become increasingly interested in machine learning, and the use of it to solve information management and retrieval problems.

Having first been exposed to the field as part of my MSc study (I took electives in information retrieval and AI / neural networks), I've continued to improve my knowledge and experience of machine learning techniques and theory, through both my professional work, and through self-directed study. While the AI classes I took at UCL were a broad introduction to artificial intelligence in general and focused predominantly on neural networks and classical AI, I've since become increasingly interested in machine learning specifically, and particularly in statistical approaches to machine learning.

Over the past three years I've taken online courses in machine learning from Stanford University / Coursera (a practical introduction to the field lead by Andrew Ng), \footnote{\url{https://www.coursera.org/course/ML}} Caltech (Yaser Abu-Mostafa's more rigorous theoretical treatment of the subject, grounded in statistical learning theory)\footnote{\url{http://work.caltech.edu/telecourse.html}}, and several undergraduate mathematics and statistics courses with the Open University (including linear algebra, real analysis, and Bayesian statistics). I am confident that this has given me the necessary mathematical background to begin research into machine learning at a postgraduate level.

In addition, I've had several opportunities to incorporate my interest in ML into my professional work as a software engineer. On graduating in 2011, I spent six months working at a quantitative advertising business called Likely Ltd\footnote{\url{http://likely.co/}}, developing predictive analytics systems incorporating natural-language processing, topic modelling, and social-network analysis components. I also spent several months working (with a friend) on a personal project with a substantial ML component; an online menu-advertising tools for specialist beer bars\footnote{\url{http://inourcellar.com}} which uses techniques from topic modelling to classify and recommend different styles of beer to customers.

Subsequently, I've worked as a freelance engineer for a variety of clients, which has also afforded many further opportunities to apply techniques from machine learning to my everyday work. In particular, I spent several months at the start of this year working with BBC Research and Development on search tools within their programme archive. While there, I was exposed to a number of interesting search and information discovery challenges encountered by users of their archive. In essence, researchers and programme makers use the archive to retrieve broadcast footage related to a highly specific topic of interest, in which they themselves are generally expert. The set of possible topics is very large, and covers a wide variety of general subject areas. Therefore, it is difficult to predict which topics will be queried ahead of time. It occurred to me that it would make sense to leverage the user's own knowledge to train a topic model to aid in their retrieval task, and after discovering the field of \textit{active learning} in the ML literature, it appeared that it might offer a principled strategy for doing so. The result of this preliminary investigation is the research proposal, attached, which I hope you consider to be an interesting and worthwhile proposal for postgraduate study. If successful in my application, I will be seeking funding from the EPSRC and/or industry.
\end{document}
