\documentclass[a4paper, 11pt]{article}
\title{PhD thesis proposal: Active learning for semantic annotation of large media archives, in the presence of structured background knowledge}
\author{Tim Cowlishaw}
\usepackage{verbatim,titlesec,lmodern}
\usepackage[T1]{fontenc}
\usepackage[backend=bibtex, citestyle=authoryear-ibid]{biblatex}
\addbibresource{/Users/tim/Documents/library.bib}
\date{\today}
\setlength{\parskip}{11pt}
\setlength{\parindent}{0pt}
\titleformat*{\subsection}{\normalsize\bfseries\itshape}
\begin{document}
\maketitle
\section{Introduction}
The BBC's \textit{Redux} (\cite{Butterworth2008}) and \textit{World Service Archive} (\cite{Raimond}) projects are large archives of audiovisual material (along with transcripts and descriptive text) with tremendous value to programme-makers and researchers both within the BBC and further afield. However, effective search and discovery within the archive is often hampered by the relative sparsity of semantic metadata describing the items in the archive, their topics and relationships. Due to the size of the archives, manual annotation of each item by professional archivists or subject experts would be prohibitively time-consuming and expensive. There are many existing approaches to automatically identifying topics and relationships in large text corpora in the machine learning literature. Howerver, traditional supervised learning approaches require an authoratitive source of ground truth labels for a subset of the corpus in the form of a training set - the manual preparation of which can also be expensive and time consuming.

We proposes an \textit{Active learning} approach (\cite{Settles2012}) to this task, which leverages the fact that the users of this system are, in general, qualified to perform this annotation themselves (for topics within their area of interest). By    interactively querying the user for ground-truth labels related to their own topic of interest, this approach would assist users with their immediate information retrieval needs by learning a relevance metric for their particular query, while also making use of the same ground-truth labels to build a general topic model of the entire archive for use in future retrieval tasks.

This project would build upon existing work in the field of active learning, natural language processing and topic modelling (reviewed in section \ref{sec:Related work}), developing a novel approach to topic-based information retrieval in large media archives with wider applications to fields such as E-discovery for the legal profession, and computer-assisted reporting. In addition, we are confident that the properties of this particular use case would serve to motivate research which would constitute an original contribution to the machine learning and natural language processing literature (discussed in more detail in section \ref{sec:Methodology}).
\section{Background}
\label{sec:Background}
The challenge of automatically annotating documents with structured semantic metadata has been addressed in previous projects by BBC Research and Development, including work on automated concept tagging and  document linking. However, to date, this work has consisted of using supervised learning models which require a training set which must be compiled by hand in advance (as in \cite{Raimond2013}), or substituted by some heuristic or external source of information which serves as a suitable proxy for ground truth (as in \cite{Raimond2012}).

Given that a large proportion of the users of the BBC Redux and World Service Archive systems are journalists, researchers and programme makers with a significant degree of knowledge around the topics of the programmes that they are seeking, it would be advantageous to leverage this knowledge in identifying topic structure within the corpus. Active learning provides a principled framework for this, by allowing a model to be incrementally trained by querying a human oracle for ground-truth classifications of documents from the corpus.

This approach has already been used with some succcess on related information retrieval problems. \cite{Lang1995} describes a use of an actively learnt model based on the minial description length principle to perform content-based filtering and recommendation on a corpus of Usenet posts, and reports an improvement in precision over TF-IDF (a modified version of which was used in \cite{Raimond2012}). In addition, \cite{McCallum1998} describes a novel use of Active learning to imporove the accuracy of a naive Bayes classifier where pre-labelled training data is sparse.

We propose to

\section{Methodology}
\label{sec:Methodology}
\section{Related work}
\label{sec:Related work}
\printbibliography
\end{document}
